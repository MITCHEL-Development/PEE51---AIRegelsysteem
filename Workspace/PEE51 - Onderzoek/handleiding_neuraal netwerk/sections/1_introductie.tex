\setlength{\headheight}{14.49998pt}
\titleformat{\section}
  {\normalfont\LARGE\bfseries\centering}
  {\thesection}{1em}{}  
%\tableofcontents

\section{introductie}
In deze handleiding wordt stap voor stap uitgelegd hoe het neuraal netwerk voor PEE51 project is opgesteld. Dit project is een onderdeel van de opleiding Elektrotechniek aan de Hogeschool Rotterdam. Het doel van deze handleiding is om stap voor stap uit te leggen hoe een neuraal netwerk ontwikkeld, getraind en geëvalueerd kan worden in Python met TensorFlow. Dit netwerk is bedoeld om het gedrag van een klassieke PID-regelaar na te bootsen of te verbeteren in een systeem waarbij een pingpongbal op een ingestelde hoogte wordt gehouden door middel van een ventilator. De aanpak richt zich op het bouwen van een werkend AI-model, met als einddoel het inzetten van dit model op een embedded platform. Daarbij ligt de nadruk op nauwkeurige en stabiele hoogtecontrole, net als bij de bestaande PID-regeling.  In deze handleiding wordt uitsluitend ingegaan op de technische implementatie van het neuraal netwerk; aspecten zoals gebruikersinterface, documentatie of demonstratieopstelling blijven buiten beschouwing. Voor meer informatie over het prestatieonderzoek en de vergelijking met de PID-regelaar, zie het bijbehorende onderzoeksrapport: \url{https://example.com/onderzoek-ai-regelaar}.

