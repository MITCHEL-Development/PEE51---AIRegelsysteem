\setlength{\headheight}{14.49998pt}
\titleformat{\section}
  {\normalfont\LARGE\bfseries\centering}
  {\thesection}{1em}{}  
%\tableofcontents

\section{datavoorbereiden}
In deze sectie wordt stap voor stap uitgelegd hoe de data wordt voorbereid voor het trainen van het neuraal netwerk. De data bestaat uit een CSV-bestand met de volgende kolommen: tijd, hoogte, snelheid, en ventilatorsnelheid. 


\begin{lstlisting}[language=Python, numbers=left, breaklines=true, basicstyle=\ttfamily\scriptsize]
CSV_PATH = '/Users/username/Documents/PEE51/data/data.csv'
CSV_DATA = pd.read_csv(CSV_PATH)
# Toon aantal samples
total_samples = len(CSV_DATA)
\end{lstlisting}

\begin{lstlisting}[language=Python, numbers=left, breaklines=true, basicstyle=\ttfamily\scriptsize]
# Selecteer features en target
features = CSV_DATA[['Setpoint (m)', 'Hoogte (m)', 'Fout', 'Fout_Integratie', 'Fout_Afgeleide']].to_numpy()
target = CSV_DATA['PWM'].to_numpy()
\end{lstlisting}


\begin{lstlisting}[language=Python, numbers=left, breaklines=true, basicstyle=\ttfamily\scriptsize]
n_input = 2  # aantal tijdstappen per voorbeeld

X, y = [], []

for i in range(len(features) - n_input):
    X.append(features[i:i + n_input])
    y.append(target[i + n_input])  # de PWM bij het volgende moment

X = np.array(X)  # shape: (samples, n_input, features)
y = np.array(y)

print("X shape:", X.shape)  # (samples, tijdstappen, features)
print("y shape:", y.shape)  # (samples,)
print("X:", X)  # (samples, tijdstappen, features)
print("y:", y) 
\end{lstlisting}
