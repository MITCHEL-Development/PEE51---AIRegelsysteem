In dit onderzoek wordt onderzocht of een neuraal netwerk het gedrag van een klassieke PID-regelaar kan nabootsen of zelfs verbeteren binnen een systeem waarin een pingpongbal op een constante hoogte wordt gehouden door middel van een ventilator. De huidige opstelling op de Hogeschool Rotterdam gebruikt een microcontroller en een PID-regelaar om de hoogte van de bal te regelen. Het afstemmen van deze regelaar is echter gevoelig voor omgevingsveranderingen en vereist handmatige tuning. Met de opkomst van machine learning rijst de vraag of een zelflerend model betere prestaties kan leveren. In dit project is een neuraal netwerk ontworpen, getraind en getest in een gesimuleerde omgeving. Het model is geëvalueerd op nauwkeurigheid, stabiliteit en generaliseerbaarheid, en vervolgens vergeleken met de klassieke PID-regelaar. De resultaten tonen aan dat het neurale netwerk het gedrag van de PID-regelaar goed kan benaderen en in sommige gevallen zelfs verbeterde prestaties levert, mits correct getraind. Hoewel de uiteindelijke implementatie op embedded hardware buiten de scope van dit onderzoek valt, is bij de modelkeuze wel rekening gehouden met beperkingen van embedded systemen. Dit onderzoek biedt waardevolle inzichten in het toepassen van kunstmatige intelligentie voor regeltechniek in embedded omgevingen en dient als opstap voor verdere integratie van slimme regelaars in praktische toepassingen.