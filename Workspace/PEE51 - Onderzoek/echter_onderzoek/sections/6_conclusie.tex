\setlength{\headheight}{14.49998pt}
\titleformat{\section}
  {\normalfont\huge\bfseries\centering}
  {\thesection}{1em}{}  

\vspace{0.2cm}
{\color{gray}\hrule}
\section{Conclusie en Aanbevelingen}
% 6. Conclusie en aanbevelingen (nog niet in pdf aanwezig)

% Doel: Beantwoord de hoofdvraag en geef advies.

% Moet bevatten:
% 	•	Kort antwoord op de hoofdvraag.
% 	•	Samenvatting van belangrijkste bevindingen.
% 	•	Aanbevelingen voor praktijk of toekomstig onderzoek.
% 	•	Eventueel suggesties voor daadwerkelijke implementatie.



\subsection{Beantwoording van de hoofdvraag}
\textbf{Is het mogelijk om een PID-regelaar te vervangen door een AI?} \\
Uit de resultaten van dit onderzoek blijkt dat een neuraal netwerk het gedrag van een PID-regelaar succesvol kan nabootsen. De AI-regelaar presteert vergelijkbaar met de traditionele PID in een gesimuleerde omgeving.

\vspace{0.5em}

\subsection{Samenvatting van de belangrijkste bevindingen}
\begin{itemize}
  \item Een \textbf{1D Convolutioneel Neuraal Netwerk (1D CNN)} bleek het meest geschikt voor deze toepassing.
  \item De prestaties van het model kwamen sterk overeen met die van de PID-regelaar.
  \item Het model is compact genoeg voor implementatie op een embedded systeem.
  \item De keuze voor een eenvoudige netwerkarchitectuur leidt zowel tot snelheid als robuustheid.
  \item Validatie met gesimuleerde data toont aan dat de AI in staat is tot stabiele en nauwkeurige regeling.
\end{itemize}



\subsection{Aanbevelingen voor implementatie}
Hoewel TensorFlow Lite ondersteuning biedt voor embedded toepassingen, bleek de EIQ-software van NXP primair gericht op beeldherkenning. Voor praktische implementatie is het daarom aanbevolen om de meegeleverde voorbeeldcode van de NXP FRDM-MCXN947 als basis te gebruiken en deze iteratief, met bijvoorbeeld trunk based development, uit te breiden met het getrainde model. Zo blijft het systeem controleer- en schaalbaar.

\subsection{Suggesties voor vervolgonderzoek}
Voor vervolgonderzoek wordt het volgende aanbevolen:
\begin{itemize}
  \item Het model fysiek implementeren op de microcontroller en de real-time prestaties evalueren.
  \item Verschillende vormen van trainingsdata vergelijken (reëel vs. simulatie).
  \item De impact van verstoringen, ruis en sensorgebreken op de robuustheid van het model analyseren.
  \item Experimenteren met andere netwerkarchitecturen, zoals LSTM's, mits de hardware dit toelaat.
\end{itemize}



