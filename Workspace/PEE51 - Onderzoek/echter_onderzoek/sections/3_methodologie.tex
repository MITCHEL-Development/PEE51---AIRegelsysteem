\setlength{\headheight}{14.49998pt}
\titleformat{\section}
  {\normalfont\huge\bfseries\centering}
  {\thesection}{1em}{}  

\vspace{0.2cm}
{\color{gray}\hrule}
% 3. Methode (nog niet in pdf aanwezig)

% Doel: Beschrijf hoe jullie het onderzoek uitgevoerd hebben.

% Moet bevatten:
% 	•	Hoe is data verzameld (bijv. via simulatie of loggen van PID-gedrag)?
% 	•	Hoe is het neuraal netwerk opgezet, getraind en gevalideerd?
% 	•	Welke tools/software zijn gebruikt?
% 	•	Beschrijving van de testopstelling of simulatieomgeving.

\section{Methode}

In dit hoofdstuk wordt beschreven op welke manier het onderzoek is uitgevoerd. De focus ligt op de software-omgeving, de manier van dataverzameling, de opbouw en training van het neuraal netwerk, en de testomgeving waarin het model is geëvalueerd.

\subsection{Software en tools}

Voor zowel de simulatie van de pingpongbal-opstelling als het opzetten en trainen van het neuraal netwerk is gebruik gemaakt van Python. De volgende libraries en tools zijn ingezet:
\begin{itemize}
    \item \textbf{NumPy} en \textbf{Pandas}: voor het verwerken en analyseren van data;
    \item \textbf{Matplotlib} voor visualisatie van resultaten;
    \item \textbf{TensorFlow} en \textbf{Keras}: voor het bouwen, trainen en valideren van het neuraal netwerk;
    \item \textbf{Scikit-learn}: voor data preprocessing en evaluatiemethoden;
    \item \textbf{Jupyter Notebook}: voor het ontwikkelen en testen van de code in iteraties.
\end{itemize}

\subsection{Dataverzameling}

De trainingsdata voor het neuraal netwerk is verzameld door een simulatie van de bestaande PID-regelaar. In deze simulatie werd het gedrag van de PID-regelaar gelogd terwijl deze een pingpongbal op verschillende hoogtes stabiel probeerde te houden. Hierbij zijn gegevens zoals tijd, doelhoogte, werkelijke hoogte, fout, fout intergratie, fout afgeleide en bijbehorende PWM-waarden opgeslagen.

De data werd vervolgens gesplitst in een trainingsset (80\%) en een validatieset (20\%).

\subsection{Neuraal netwerk opzetting}

Op basis van de kenmerken van het systeem en de vereisten voor embedded implementatie is gekozen voor een 1D Convolutioneel Neuraal Netwerk (1D CNN). Dit type netwerk is efficiënt in het herkennen van patronen in sequentiële data en heeft relatief lage rekenvereisten.

De architectuur van het netwerk bestaat uit:
\begin{itemize}
    \item Een inputlaag met vensters van opeenvolgende foutwaarden;
    \item Een of meerdere 1D convolutionele lagen met ReLU-activatie;
    \item Een flatten-laag gevolgd door één dense laag;
    \item Een outputlaag die de benodigde PWM-waarde voorspelt.
\end{itemize}

\subsection{Trainingsstrategie en hyperparameters tuning}
Het model is getraind met behulp van backpropagation en de Adam optimizer. De volgende hyperparameters zijn getest en afgestemd:
\begin{itemize}
    \item Learning rate: $0.001$, $0.0005$, $0.0001$;
    \item Batch size: 32 en 64;
    \item Aantal epochs: 50–200;
    \item Rechthoekige activatievensters: lengte 5–20 samples.
\end{itemize}

De performance van het model werd tijdens de training geëvalueerd met behulp van de mean squared error (MSE). Early stopping werd toegepast om overfitting te voorkomen.

\subsection{Testopstelling of Simulatieomgeving}

De opstelling is volledig gesimuleerd in Python, waarbij de dynamiek van de pingpongbal in een verticale buis werd gemodelleerd. Hierbij werd rekening gehouden met de krachten op de bal (zwaartekracht, luchtdruk door de ventilator) en de vertraging in het systeem. Zowel de PID-regelaar als het neuraal netwerk werd getest in dezelfde simulatieomgeving zodat een eerlijke vergelijking kon worden gemaakt.

Het gedrag van beide regelstrategieën werd geëvalueerd op responsietijd, stabiliteit, overshoot, en robuustheid bij verstoringen.








