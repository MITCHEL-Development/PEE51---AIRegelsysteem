\begin{multicols}{2}
\tableofcontents
\section{Probleemschets}
Hogeschool Rotterdam beschikt voor de cursus Digitale Systemen (DIS10) over een opstelling waarbij een pingpongbal op een ingestelde hoogte wordt gehouden. Deze opstelling bestaat uit een verticale buis met onderaan een ventilator. Door de luchtdruk van de ventilator wordt de bal omhoog geblazen. Met behulp van een microcontroller wordt de ventilator aangestuurd, zodat de bal op een gewenste hoogte blijft zweven. De regeling van de ventilatorsnelheid gebeurt momenteel met behulp van een PID-regelaar (Proportioneel-Integrerend-Differentieel), waarbij de afwijking tussen de gewenste en werkelijke hoogte continu wordt bijgestuurd.

Hoewel deze opstelling functioneert zoals bedoeld, is het instellen van de PID-parameters vaak een handmatig proces. Het vergt tijd om het systeem goed af te stemmen om het uiteindelijk stabiel te laten reageren op hoogteveranderingen. Daarnaast kunnen veranderingen in de omgeving (zoals temperatuur of luchtvochtigheid) de prestaties van de PID-regelaar beïnvloeden.

De begeleidende docent heeft de vraag gesteld of deze traditionele regelmethode vervangen kan worden door een benadering gebaseerd op machine learning. Machine learning biedt de mogelijkheid om op basis van data te leren hoe het systeem zich moet gedragen. In dit project wordt onderzocht of het mogelijk is om een machine learning model te ontwikkelen dat het gedrag van de PID-regelaar kan nabootsen of zelfs verbeteren.

Dit onderzoek is relevant omdat het laat zien hoe klassieke regeltechniek gecombineerd of vervangen kan worden door moderne, zelflerende systemen. Daarnaast sluit het aan bij actuele ontwikkelingen binnen de techniek en biedt het studenten inzicht in zowel embedded systemen als kunstmatige intelligentie. Uiteindelijk is het doel om een demonstratie-opstelling te creëren die gebruikt kan worden tijdens bijvoorbeeld open dagen, om zo op een toegankelijke manier beide technieken te tonen.


Een neuraal netwerk\footnote{Een \textit{neuraal netwerk} is een model geïnspireerd op het menselijk brein.} wordt getraind met data.

\end{multicols}