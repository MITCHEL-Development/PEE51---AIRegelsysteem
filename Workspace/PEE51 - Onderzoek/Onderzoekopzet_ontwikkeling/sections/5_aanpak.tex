\vspace{0.3cm}
{\color{gray}\hrule}
\section{Aanpak}
\vspace{0.3cm}
{\color{gray}\hrule}

\subsection{Data}
Voor dit deelonderzoek wordt trainingsdata verzameld in een gesimuleerde omgeving waarin de traditionele PID-regelaar actief is. De verzamelde gegevens bestaan uit:

\begin{itemize}
    \item De gewenste hoogte (setpoint)
    \item De gemeten werkelijke hoogte van de pingpongbal
    \item De door de PID-regelaar gegenereerde aanstuurwaarde (bijv. PWM-percentage)
\end{itemize}

Deze data worden gestructureerd als input-output-paren waarmee het neuraal netwerk wordt getraind. De variatie in setpoints en omgevingscondities wordt meegenomen om het model robuust te maken.

\subsection{Methode}
De volgende methoden worden toegepast binnen het onderzoek:
  
\begin{enumerate}
    \item Analyse van de bestaande PID-regelaar in de simulatieomgeving.
    \item Verzamelen en opschonen van trainingsdata.
    \item Selectie van een geschikte netwerkarchitectuur (bijv. feedforward netwerk).
    \item Opzetten van de trainingsomgeving in Python (bijv. TensorFlow of PyTorch).
    \item Trainen van het model met gebruik van de gestructureerde data.
    \item Evaluatie van de prestaties op basis van simulatie-uitvoer, zoals foutmarge, stabiliteit en convergeersnelheid.
    \item Exporteren van het getrainde model voor overdracht naar Team 2.
\end{enumerate}
  
Tijdens het trainen worden verschillende hyperparameters getest, zoals het aantal lagen, neuronen per laag, activatiefuncties en optimalisatiemethoden (zoals Adam of SGD).

\subsection{Planning}

  
De werkzaamheden van Team 1 zijn verdeeld over de looptijd van het project zoals hieronder weergegeven:
  
\begin{table}[ht!]
\centering
\begin{tabular}{|l|p{10cm}|}
\hline
\textbf{Week} & \textbf{Activiteit} \\
\hline
Week 1–2 & Analyse van PID-regelaar en opzetten simulatieomgeving \\
Week 3–4 & Verzamelen en opschonen van trainingsdata \\
Week 5–6 & Ontwerp en training van het neuraal netwerk \\
Week 7   & Validatie van het model en prestatievergelijking met PID \\
Week 8   & Finaliseren van het model, documentatie en overdracht aan Team 2 \\
Week 9–10 & Ondersteuning bij presentatie en reflectie op de resultaten \\
\hline
\end{tabular}
\caption{Planning Team 1 gedurende het project}
\end{table}
