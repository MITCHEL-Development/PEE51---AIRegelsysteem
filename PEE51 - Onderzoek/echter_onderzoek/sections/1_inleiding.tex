\setlength{\headheight}{14.49998pt}
\titleformat{\section}
  {\normalfont\LARGE\bfseries\centering}
  {\thesection}{1em}{}  
%\tableofcontents

\section{Inleiding}
Hogeschool Rotterdam beschikt over de cursus Digitale Systemen (DIS10) over een opstelling waarbij een pingpongbal op een ingestelde hoogte wordt gehouden. Deze opstelling bestaat uit een verticale buis met onderaan een ventilator. Door de luchtdruk van de ventilator wordt de bal omhoog geblazen. Met behulp van een microcontroller wordt de ventilator aangestuurd, zodat de bal op een gewenste hoogte blijft zweven. De regeling van de ventilatorsnelheid gebeurt momenteel met behulp van een PID-regelaar (Proportioneel-Integrerend-Differentieel), waarbij de afwijking tussen de gewenste en werkelijke hoogte continu wordt bijgestuurd.

Hoewel deze opstelling functioneert zoals bedoeld, is het instellen van de PID-parameters vaak een handmatig proces. Het vergt tijd om het systeem goed af te stemmen om het uiteindelijk stabiel te laten reageren op hoogteveranderingen. Daarnaast kunnen veranderingen in de omgeving (zoals temperatuur of luchtvochtigheid) de prestaties van de PID-regelaar beïnvloeden.

De begeleidende docent heeft de vraag gesteld of deze traditionele regelmethode vervangen kan worden door een benadering gebaseerd op machine learning. Machine learning biedt de mogelijkheid om op basis van data te leren hoe het systeem zich moet gedragen. In dit project wordt onderzocht of het mogelijk is om een machine learning model te ontwikkelen dat het gedrag van de PID-regelaar kan nabootsen of zelfs verbeteren.

Dit onderzoek is relevant omdat het laat zien hoe klassieke regeltechniek vervangen kan worden door moderne, zelflerende systeem. Daarnaast sluit het aan bij actuele ontwikkelingen binnen de techniek en biedt het studenten inzicht in zowel embedded systemen als kunstmatige intelligentie. Uiteindelijk is het doel om een demonstratie-opstelling te creëren die gebruikt kan worden tijdens bijvoorbeeld open dagen, om zo op een toegankelijke manier beide technieken te tonen.

Het doel van dit onderzoek is om te onderzoeken of een neuraal netwerk het gedrag van een PID-regelaar effectief kan nabootsen of verbeteren in een systeem waarbij een pingpongbal op een ingestelde hoogte wordt gehouden. Hierbij wordt een model ontwikkeld, getraind en geëvalueerd in een gesimuleerde omgeving. Uiteindelijk moet het model geschikt zijn voor implementatie op een embedded platform, met als doel een stabiele en nauwkeurige hoogtecontrole te realiseren. Het onderzoek richt zich op zowel de technische haalbaarheid als de vergelijking met de bestaande PID-regelaar. we gaan in dit onderzoek niet in op de implementatie van een neuraal netwerk in een embedded omgeving, maar er wordt wel rekening gehouden met dat het uiteindelijk in een embedded omgeving geimplementeerd moet worden. In dit onderzoek wordt de nadruk gelegd op of het mogelijk is om een neuraal netwerk het PID-regelaar gedrag te laten nabootsen en hoe dit zich verhoudt tot de traditionele PID-regelaar.

Voor dit onderzoek zijn er een aantal onderzoeksvragen geformuleerd die uiteindelijk de hoofdvraag van dit onderszoek moeten beantwoorden:
\begin{enumerate}
  \item Hoe werkt een PID-regelaar en hoe wordt deze toegepast in de huidige opstelling?
  \item Welke soorten neurale netwerken zijn geschikt voor regeltoepassingen in embedded systemen?
  \item Hoe kan trainingsdata verzameld worden om het gedrag van de PID-regelaar na te bootsen?
  \item Hoe presteert het getrainde neurale netwerk vergeleken met de traditionele PID-regelaar?
  \item Wat zijn de beperkingen van het gebruik van een neuraal netwerk in een embedded omgeving?
\end{enumerate}

aan het einden van het onderzoek zullen deze deelvragen worden beantwoord en onderbouwd, zodat de hoofdvraag kan worden beantwoord: \textit{Is het mogelijk om een PID-regelaar te vervangen door een AI?}

Voor dit onderzoek wordt zal er rekening gehouden worden met dat het neuraal netwerk op een **NXP FRDM-MCXN947 Development board** microcontroller moet komen. in het verloop zal van het onderzoek wordt er rekening gehouden met de beperkingen van deze microcontroller, zoals geheugen- en rekenkrachtbeperkingen. Dit is belangrijk omdat het uiteindelijke doel is om het getrainde model op deze microcontroller te implementeren. De NXP FRDM-MCXN947 Development board is gekozen omdat deze geschikt is voor embedded toepassingen en voldoende mogelijkheden biedt voor het uitvoeren van machine learning taken.



In dit rapport wordt eerst de benodigde theorie besproken (hoofdstuk 2), gevolgd door de gebruikte methoden (hoofdstuk 3) waarin we dit onderzoek doen. Daarna worden de resultaten gepresenteerd (hoofdstuk 4) en besproken (hoofdstuk 5). Het rapport sluit af met conclusies en aanbeveling (Hoofdstuk 6).

Een neuraal netwerk\footnote{Een \textit{neuraal netwerk} is een model geïnspireerd op het menselijk brein.} wordt getraind met data.

