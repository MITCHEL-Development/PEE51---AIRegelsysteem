In dit onderzoek wordt onderzocht of een neuraal netwerk het gedrag van een PID-regelaar kan nabootsen of verbeteren in een systeem waarbij een pingpongbal op een ingestelde hoogte wordt gehouden. De huidige regeling werkt met een PID-regelaar, maar het afstellen hiervan is tijdrovend en gevoelig voor omgevingsfactoren. Door een machine learning-model te trainen met data van het systeem, wordt onderzocht of een alternatief regelalgoritme mogelijk is. Er wordt niet ingegaan op de implementatie van het netwerk in een embedded omgeving, maar er wordt wel rekening gehouden met de beperkingen van de NXP FRDM-MCXN947 microcontroller, waarop het model uiteindelijk moet draaien. Het onderzoek richt zich op de technische haalbaarheid en de prestatievergelijking tussen het neurale netwerk en de traditionele PID-regelaar.