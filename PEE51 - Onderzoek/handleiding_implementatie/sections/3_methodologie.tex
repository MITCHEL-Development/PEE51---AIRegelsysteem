\vspace{0.2cm}
{\color{gray}\hrule}
\section{Methode}
Een van de eerste stappen in dit onderzoek is het verkrijgen van trainingsdata voor het neurale netwerk. Hiervoor zijn er een aantal methoden en technieken nodig om de data te verzamelen en het model te trainen.


\subsection{Simulatieomgeving en uitgangspunten}
Een van de uitdagingen bij dit onderzoek is 




Een van de uitdagingen bij het onderzoeken van de mogelijkheid om een PID-regelaar te vervangen door een AI is het creëren van een geschikte simulatieomgeving. Deze omgeving moet de dynamiek van het systeem nauwkeurig nabootsen en voldoende gegevens genereren voor het trainen van het neurale netwerk. De uitgangspunten voor de simulatie zijn als volgt:
\begin{itemize}
  \item De simulatie moet de werking van de huidige PID-regelaar nabootsen, inclusief de dynamiek van de pingpongbal en de ventilator.
  \item De simulatie moet in staat zijn om verschillende scenario's te genereren, zoals veranderingen in de gewenste hoogte, verstoringen en variaties in de omgeving.
  \item De simulatie moet voldoende gegevens genereren om het neurale netwerk te trainen, inclusief invoer- en uitvoerwaarden die representatief zijn voor de werkelijke situatie. 
  \item De simulatie moet eenvoudig aan te passen zijn, zodat verschillende parameters van het systeem kunnen worden getest en geoptimaliseerd.
  \item De simulatie moet compatibel zijn met de NXP FRDM-MCXN947 Development board, zodat het getrainde model later eenvoudig kan worden geïmplementeerd op de microcontroller.
  \item De simulatie moet gebruik maken van een programmeertaal en bibliotheken die geschikt zijn voor zowel de simulatie als de implementatie op de microcontroller, zoals Python met TensorFlow of Keras.
\end{itemize}

\subsection{verzameling en trainingsdata}
Een van de uitdagingen van dit onderzoek is het verzalemeln van voldoende en representatieve trainingsdata voor het neurale netwerk. De trainingsdata moet de dynamiek van het systeem nauwkeurig weergeven en variaties bevatten die het model in staat stellen om te generaliseren naar nieuwe situaties. De volgende stappen worden ondernomen om de trainingsdata te verzamelen:

\begin{itemize}
  \item \textbf{Simulatie van de opstelling:} De simulatieomgeving wordt gebruikt om het gedrag van de PID-regelaar en de pingpongbal te modelleren. Hierbij worden verschillende scenario's gecreëerd, zoals veranderingen in de gewenste hoogte, verstoringen en variaties in de omgeving.
  \item \textbf{realtime input en output data genereren van de opstelling:} Voor elk scenario worden invoerwaarden (zoals de gewenste hoogte en eventuele verstoringen) en uitvoerwaarden (zoals de PWM waarde) gegenereerd. Deze waarden vormen de basis voor de trainingsdata.
\end{itemize}

\subsection{Netwerkarchitectuur en configuratie}
Voor de 

\subsection{Trainingsstrategie en hyperparameters tuning}

\subsection{Validatiemethoden en evaluatiecriteria}


