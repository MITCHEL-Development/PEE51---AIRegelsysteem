\setlength{\headheight}{14.49998pt}
\titleformat{\section}
  {\normalfont\LARGE\bfseries\centering}
  {\thesection}{1em}{}  

\vspace{0.3cm}
{\color{gray}\hrule}
\section{Doel van het onderzoek}
\vspace{0.3cm}
{\color{gray}\hrule}
\begin{multicols}{2}
\subsection{Doelstellingen}
Het doel van dit deel van het onderzoek is om een neuraal netwerk te ontwikkelen dat in staat is het gedrag van een PID-regelaar na te bootsen in een simulatieomgeving. Door middel van het verzamelen van trainingsdata, het kiezen van een geschikte netwerkarchitectuur en het trainen van het model, wordt onderzocht of het netwerk vergelijkbare of betere prestaties kan leveren dan de traditionele regelaar. Het uiteindelijke doel is om een model op te leveren dat klaar is voor implementatie op een embedded systeem, waarbij de nadruk ligt op nauwkeurigheid, stabiliteit en reproduceerbaarheid.
\subsection{Belanghebbenden}
De belangrijkste belanghebbenden in dit deelonderzoek zijn de studenten en docenten van de opleiding Elektrotechniek aan de Hogeschool Rotterdam. Voor studenten biedt dit project een leerervaring in het toepassen van kunstmatige intelligentie binnen een praktisch regelsysteem. Ze leren niet alleen over neurale netwerken, maar ook over de analyse van bestaande regelaars, datastructuren en modeltraining.

Voor docenten is dit onderzoek van waarde als onderwijsmiddel; het toont aan hoe klassieke regeltechniek gecombineerd kan worden met moderne AI-toepassingen. Daarnaast draagt het bij aan het ontwikkelen van vernieuwend lesmateriaal dat aansluit bij de actuele technologische ontwikkelingen. Ook toekomstige studenten kunnen profiteren van de uitkomsten van dit onderzoek doordat het mogelijk geïntegreerd wordt in demonstratieopstellingen of lesmodules
\end{multicols}
